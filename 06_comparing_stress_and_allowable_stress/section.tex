\section{Comparing Stress and Allowable Stress}

\newthought{now armed with the calculations} for the shaft stresses and design factor, you will need to select nodes along the shaft that you feel may be critical to its operation. This is for you to decide and to discuss within your reports.

With the nodes selected, you can start to populate the excel spreadsheet that is provided. Placing the calculations within the cells enables you to quickly perform iterations of the shafts design and clearly articulate the changes to the shafts geometry. You will be submitting the final spreadsheet so we can review your calculations within the cells. Please do not alter the format of the spreadsheet. 

\begin{table}[h!]
  \caption{Spreadsheet stress model}
  \centering
  \small
  \begin{tabular}{l | c | c c c}
    \toprule
      & Node No. & 1 & 2 & 3\\
    Node Details & Units & \\
    \midrule
    Diameter \\
    Area \\
    Second Moment of Area \\
    Second Polar Moment of Area \\
    \midrule
    Forces \\
    \midrule
    Vertical \\
    Horizontal \\
    Resultant \\
    Resultant Angle \\
    \midrule
    Bending \\
    \midrule
    Vertical \\
    Horizontal \\
    Resultant \\
    Torque \\
    \midrule
    Stresses \\
    \midrule
    Direct Stress \\
    Bending Stress \\
    Torsional Stress \\
    \midrule
    Principal Stresses \\
    \midrule
    Principal Stress 1 (+) \\
    Principal Stress 2 (-) \\
    Shear Stress \\
    \midrule
    Safety Factors \\
    \midrule
    a \\
    b \\
    c \\
    d \\
    k \\
    Nu \\
    \midrule
    UTS \\
    Allowable Stress \\
    \bottomrule
  \end{tabular}
\end{table}
